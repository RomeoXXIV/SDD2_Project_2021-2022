% Author : Alexandre Quenon
% Last update : Décembre 2020 by Ibraimovski Roméo

% % % % % % %
%  Packages %
% % % % % % %

%---Base packages
\documentclass[a4paper,12pt]{report}	% document type (article, report, book)
\usepackage[utf8]{inputenc}			% encoding
\usepackage[T1]{fontenc}			% accent
\usepackage{lmodern}				% latin font
\usepackage{appendix}				% to be able to use appendices

%---Language(s)
\usepackage[english,frenchb]{babel}	% last language = typography by default
\addto\captionsfrench{				% to change the french names of...
	\renewcommand{\appendixpagename}{Annexes}	% (default Appendices)
	\renewcommand{\appendixtocname}{Annexes}	% (default Appendices)
	\renewcommand{\tablename}{\textsc{Tableau}}	% (default \textsc{Table})
}

%---Page layout
%------> margins
	% 1st option -> geometry package
		%\usepackage[a4paper]{geometry}		% default parameters for A4
		%\usepackage[top=2in, bottom=1.5in, left=1in, right=1in]{geometry}
	% 2nd option -> a4wide package
		\usepackage{a4wide}		% A4 with smaller margins (the one I've chosen)
	% 3rd option -> fullpage package
		%\usepackage{fullpage}
%------> chapter style
	% 1st option -> fncychap package
		%\usepackage[style]{fncychap}		% style = Lenny, Bjornstrup, Sonny, Conny
	% 2nd option -> customized styles
		%
%------> cover page (UMONS template)
	\usepackage[fs]{umons-coverpage}		% NEED "umons-coverpage.sty" file
	\umonsAuthor{Made by Roméo \textsc{Ibraimovski} \& Maxime \textsc{Nabli}}
	\umonsTitle{Rapport de projet}
	\umonsSubtitle{Exercice préliminaire}
	\umonsDocumentType{S-INFO-820	Projet de structures de données II}
	\umonsSupervisor{Supervised by Gauvain \textsc{Devillez}}
	\umonsDate{3e Bachelier en Sciences Informatiques\\ Année 2021-2022}

%---Numbering
\setcounter{secnumdepth}{2}			% numerotation depth (1=sec and all above)
\setcounter{tocdepth}{2}			% table of contents depth (1=sec and above)

%---Mathematics
\usepackage{amsmath}				% base package for mathematics
\usepackage{amsfonts}				% base package for mathematics
\usepackage{amssymb}				% base package for mathematics
%\usepackage{amsthm}				% theorem and proof environments
%\usepackage{cases}					% numcases environment
%\usepackage{mathrsfs}				% \mathscf command ('L' of Laplace-Transform,...)

%---Floating objects (images, tables,...)
\usepackage{float}					% better management of floating objects
\usepackage{array}					% better management of tables
\usepackage{graphicx}				% to include external images
\graphicspath{{Images/}}			% to put images in an 'Images' folder 
%\usepackage{caption}				% /!\ has priority on "memoir" class
%\usepackage{subcaption}			% subfigure and subtable environments
%\usepackage{subfig}				% \subfloat command
%\usepackage{wrapfig}				% wrapfigure environment
%\usepackage[update]{epstopdf}		% to use '.eps' files with PDFLaTeX

%---Code including
\usepackage{listings}				% general package (can be tuned)
%\usepackage[framed]{mcode}			% to include Matlab code
									% /!\ you need the "mcode.sty" file

%---Units from International System
\usepackage{siunitx}				% \SI{}{} command (units with good typography)
\DeclareSIUnit\baud{baud}			% definition of the "baud" unit
\DeclareSIUnit\bit{bit}				% definition of the "bit" unit

%---Drawing
%\usepackage{tikz}					% useful package for drawing
%\usepackage[european]{circuitikz} 	% to draw electrical circuits

%---Bibliography
\usepackage{url}					% to encore url
\usepackage[style=numeric-comp,backend=bibtex]{biblatex}
\usepackage{csquotes}				% inverted commas in references
%\bibliography{bibli}				% your .bib file

%---"hyperref" package				% /!\ it must be the last package
\usepackage[hidelinks]{hyperref}	% clickable links (table of contents,...)

%Pour écrire des algos: --------------------------------------------------------------------------
\usepackage{algorithmic, algorithm}

%Traduction en français
\floatname{algorithm}{Algorithme}
\renewcommand{\algorithmicrequire}{\textbf{Entrée :}}
\renewcommand{\algorithmicensure}{\textbf{Sortie :}}
\renewcommand{\algorithmicend}{\textbf{fin}}
\renewcommand{\algorithmicif}{\textbf{si}}
\renewcommand{\algorithmicthen}{\textbf{alors}}
\renewcommand{\algorithmicelse}{\textbf{sinon}}
\renewcommand{\algorithmicelsif}{\algorithmicelse\ \algorithmicif}
\renewcommand{\algorithmicendif}{\algorithmicend\ \algorithmicif}
\renewcommand{\algorithmicfor}{\textbf{pour}}
\renewcommand{\algorithmicforall}{\textbf{pour tout}}
\renewcommand{\algorithmicdo}{\textbf{faire}}
\renewcommand{\algorithmicendfor}{\algorithmicend\ \algorithmicfor}
\renewcommand{\algorithmicwhile}{\textbf{tant que}}
\renewcommand{\algorithmicendwhile}{\algorithmicend\ \algorithmicwhile}
\renewcommand{\algorithmicloop}{\textbf{boucle}}
\renewcommand{\algorithmicendloop}{\algorithmicend\ \algorithmicloop}
\renewcommand{\algorithmicrepeat}{\textbf{répéter}}
\renewcommand{\algorithmicuntil}{\textbf{jusqu'à}}
\renewcommand{\algorithmicprint}{\textbf{imprimer}}
\renewcommand{\algorithmicreturn}{\textbf{retourner}}
\renewcommand{\algorithmictrue}{\textbf{vrai}}
\renewcommand{\algorithmicfalse}{\textbf{faux}}
\renewcommand{\and}{\textbf{et}}


%Enleve la numérotation des algorithmes
\makeatletter
\renewcommand{\fnum@algorithm}{\fname@algorithm}
\makeatother
%------------------------------------------------------------------------------------------------

% % % % % % %
% Document	%
% % % % % % %

\begin{document}

% Front matter
% ------------
%\frontmatter			% with class "book" only

	\umonsCoverPage		% produce the cover page with UMONS and your Faculty logo
	
	\pagenumbering{roman}	% if you don't use the class "book"
	
	\begin{abstract}	% environment adapted to write the abstract
		...
	\end{abstract}
	
	\clearpage			% to start the toc on a new page
	\tableofcontents
	%\clearpage			% to start the lof on a new page
	%\listoffigures
	%\clearpage			% to start the lot on a new page
	%\listoftables
	

% Main matter
% -----------
%\mainmatter			% with class "book" only

	\clearpage			% if you don't use the class "book"
	\pagenumbering{arabic}
	
	{\section*{1. Introduction}}
	\addcontentsline{toc}{chapter}{1. Introduction}	% to add in the toc
	
	\begin{algorithm}
	\caption{Calculate $y = x^n$}
	\label{Modele pour un algo}
	\begin{algorithmic} [1]
	\REQUIRE $n \geq 0 \vee x \neq 0$
	\ENSURE $y = x^n$
	\STATE $y \gets 1$
	\IF{$n < 0$}
	\STATE $X \leftarrow 1 / x$
	\STATE $N \leftarrow -n$
	\ELSE
	\STATE $X \leftarrow x$
	\STATE $N \leftarrow n$
	\ENDIF
	\WHILE{$N \neq 0$ \and $N \neq 0$}
	\IF{$N$ is even}
	\STATE $X \leftarrow X \times X$
	\STATE $N \leftarrow N / 2$
	\ELSE[$N$ is odd]
	\STATE $y \leftarrow y \times X$
	\STATE $N \leftarrow N - 1$
	\ENDIF
	\ENDWHILE
	\end{algorithmic}
	\end{algorithm}
			
	{\section*{Conclusion}}
		\addcontentsline{toc}{chapter}{Conclusion}	% to add in the toc	
	
\noindent Raptim igitur properantes ut motus sui rumores celeritate nimia praevenirent, vigore corporum ac levitate confisi per flexuosas semitas ad summitates collium tardius evadebant. et cum superatis difficultatibus arduis ad supercilia venissent fluvii Melanis alti et verticosi, qui pro muro tuetur accolas circumfusus, augente nocte adulta terrorem quievere paulisper lucem opperientes. arbitrabantur enim nullo inpediente transgressi inopino adcursu adposita quaeque vastare, sed in cassum labores pertulere gravissimos.
	
	% Appendices
	%\clearpage
	%\appendix		% Changes the numbering of chapters to an alphabetic form and
					% also changes the names of chapters from \chaptername 
					% to the value of \appendixname.
	%\addappheadtotoc	% Makes an entry in the ToC
	%\appendixpage	% Generates a part-like page with the title given by
					% the value of \appendixpagename. 
	
	%\chapter{The first appendix}
	
		


% Back matter
% -----------
%\backmatter			% with class "book" only
	
	% Bibliography
		%\clearpage			% if it does'nt start on a new page
		%\phantomsection	% if any problem with the reference in the toc
	\printbibliography
	%\addcontentsline{toc}{chapter}{Bibliographie}

\end{document}
